\chapter{Discussion}
This first section of the discussion will focus on the evaluation chapter. Next the discussion will look into the main points that can be used to improve the design. The chapter will then discuss some of the choices made along the way for each control scheme implemented and look at what alternatives there might exist. Finally the chapter discuss some of the technical difficulties that the joystick control scheme experienced, as well as look into what could have made the test level better suited for testing navigation.

\section*{Test related discussion}
The testing of this implementation, have a few points that should change, specifically three points have been identified that has had a negative impact on the test. These are:

\begin{itemize}
\item Test environment
\item Tested users can be classified as gamers
\item Competition
\end{itemize}

\subsection*{Test environment}
During the testing of this prototype the environment was not taken into account as it was done in the team's group room. This introduced some problems that should not have been present, for instance: Two participants were brought in to test for the team, these were not separated during their individual tests and the participants were not told not to talk during the test. This might have affected their responses as well as their completion time. This could have been fixed by having the participants do the test in one room and the card sorting in another. If the tests had been carried out in two rooms, one room could have held a one-way mirror where the testers could have video taped and observed the test participants from. This could have allowed an in-depth analysis of the emotional responses from the participants. 

\subsection*{tested users are gamers} \label{TestedUsersAreGamers}
An issue that was not taken into account before the test was carried out, was the fact that many of our test participants were not only digital natives but also gamers.\footnote{people  who play video games on any platform} This meant that the target group was not well represented. They may have had more skills than anticipated which may have affected the outcome of the card sorting. E.g. that the buttons were boring, that they are used to a joystick and have a smaller knowledge gap than the target group normally would. Therefore the current knowledge point may have been significantly closer to the target knowledge point for gamers. 
It should also be noted that while our test participants found the button controls the most familiar, a person who has no preconceptions about how navigation should work, might find the gyroscopic controls easier to learn, as it relates more to the real world. 
\subsection*{Anomalies in the test data}
When looking at the data gathered from the tests, the data of two test subjects stands out as anomalies. While the joystick is generally slower than the other two control schemes, there was one test participant who spent nearly twice as long as the second slowest test participant. 
The deviation from the norm in this test was so large that it single handedly increased the average time by close to 25\%.
Another anomaly we noticed was with the gyroscope. One of our test participants ran into some technical issues during the test, to which we could attribute his overwhelmingly negative comments. This test participant accounted for 40\% of the gyroscopes negative comments.
One must note, that the first of these anomalies can be eliminated since it's part of our quantitative data, while the other can not since it's qualitative, even though it has a big impact on the data.
\subsection*{Test bias}
The test results may have been biased in favour of the button controls, since the arrows on the buttons are actually descriptive of what they do. Opposed to the gyroscope which does not have anything visual indicators for its gyroscopic interaction. 
\section*{Competition}
The test was designed as a competition in hopes of motivating the user to do their best i.e. as fast as they could. This worked well during the test but it later showed that the test participants might have been too focused on the competition rather than the navigation. This could also be because they were not instructed properly before the card sorting. It affected the card sorting results, as many chose cards like "fun" or "engaging" and reasoned it with the fact that it was fun because it was a competition. It was quickly noticed a trend in this and made sure to instruct the subsequent test participants that the comments should be related to the control and the controls only. 
We also chose to disregard the comments related to the test's premises.

\section*{Test data to redesign}
In order to test whether the button controls are actually the most efficient, another test should be done where the test participants get a chance to play around with the controls before being timed. This should eliminate, or at least reduce the time needed to learn the controls, allowing the test to focus only on efficiency.
While the joystick controls received mostly negative feedback there were some positive comments, 
particularly regarding the fact that it alone allowed the user to move sideways instead of just back 
and forward. This feature is something that shall be brought into the next iteration. The fact that the users would like to be able to move sideways can relate to the fact that a lot of the test participants are gamers as mentioned on page \ref{TestedUsersAreGamers}. 

\section*{Control schemes choices}
Analysis of the state of the art gave an understanding of which controls are most common and should be the most familiar to users with digital knowledge.

\subsection*{Limited Control Designs} 
The control scheme designs were limited to three. These were implemented, as they were the ones that initially seemed to minimize the knowledge gap through familiarity. However, design alternatives should not be overlooked, as possible combinations of the initial control schemes could bring new interesting outcomes. There is a wide variety of possible alternative ways of controlling a 3D environment. In the Redesign section possible alternative control designs will be presented.

\subsection*{Joystick - not calibrated nor adaptive}\label{DiscussionJoystick}
Testing showed that there is one main problem with the joystick control's implementation. Graphically the joystick is displayed in the center of its boundaries which are represented as a white circle. But it was implemented as though the actual center is in the corner of the joystick. The current state of the joystick controls only works when the fingertip is placed in the center and dragged. Some of the users placed the finger on the side of the control field assuming that the camera would still turn. There are different approaches in solving this problem and a few of them will be discussed in the redesign chapter. 

\subsection*{Gyroscopic approach and the alternatives}
When designing the gyroscopic control scheme, different approaches were discussed. The approach taken was to eliminate all the navigation from the screen that could be done in an alternative way (by moving the device). This control scheme was the one with least amount of design elements on the screen.
As much as this control scheme could be a working approach to navigate in virtual space, some general things have to be considered. For instance, moving around for a longer period of time or being in an environment that is not optimal for moving around. Although the most of the test participants did not point that out, it may be an issue in the future if the problem area expands.
Alternatively, an additional version of a control scheme that uses the gyroscope sensor, could go into the next iteration is discussed in the redesign chapter.

\section*{Technical Difficulties}
When the team decided on implementing the three control schemes there was no real way of estimating how difficult implementing each control scheme would be. This meant that in order for the prototypes to be ready for testing, the implementation was not planned as rigidly as it could have been. This lead to the joystick having more technical difficulties than the other two control schemes. Furthermore, the joystick implementation is composed of both the teams own assets and some of the standard assets provided by Unity. This meant that for the team to be able to modify the code, the provided code from unity had to be analysed and understood. This difficulty has caused some negative test results for the joystick as mentioned in the evaluation chapter (see page \pageref{joystickEvaluationResults}). 

\section*{3D Level design}
The test level was effective and helped to make tests with useful results. However there was one small aspect that needs to be addressed. The plan was that participants would go through the level as fast as possible. However, the level was designed for further possible design iterations where each improved control scheme would be tested for how efficient the user could move around. This affected the test participants, even if they were told to ignore the text and other signs and just go through the test as fast as possible. 

Other aspects that caused discussion was that some users had a problem when with obstacles and loosing time. This was because the implementation caused inaccurate user movement. 
This might cause some negative feedback in the card sorting as mentioned in evaluation. However this does not affect the comparison between the different schemes, since all the controls were calibrated to match each other, speed-wise and the 3D collider that they use. 

\section*{Iterative design process}
The process that was used for this project was focused on information gathering and creating a product based on this information.
Instead of focusing all the effort on finding research the design process could have been more user 
centred as mentioned in section \ref{UXUserCentred}.
This would mean creating low-fi prototypes and doing quick and dirty testing from the beginning. This 
would allow the team to go through several iterations where we heighten the fidelity and eliminate errors 
where the focus of the project is being narrowed down with each iteration. Doing several iterations 
would have allowed us to catch technical issues like the joystick controls, early on 
in the process, allowing for several tests with a decreasing amount of errors.
\section*{Answering the FPS}
In this section it will be discussed if the Final Problem Statement has been answered and to what extent. The final problem states: ”How does non-traditional interaction using the concept of familiarity help minimize the knowledge gap for users with digital wisdom, for navigating in a 3D environment?”. To evaluate whether the statement answered, the most crucial aspects of the report will be concluded upon. 
Most of the test participants were either gamers or power-users of computers and/or smart phones and as such is technically included in the target group, however they are not representative of the target group as a whole. 
The test was developed with the mindset that faster completion times would equal a more familiar control scheme. This was the reason that the test participants were not allowed to try out the controls before being asked to complete the obstacle course. The test did not have a clear way of figuring out when exactly each test participant had a good understanding of how the controls worked. This limits the validity of the test for finding out if the FPS was answered.
After concluding upon evaluation it showed that the users were quick to master all of the navigation methods but buttons proved to be most efficient. This was because buttons are immediately recognizable by almost everyone and that it is also an interaction which the users are familiar with. This seems to support the idea that the familiarity concept minimizes the knowledge gap. On top of that the buttons are self explanatory since the buttons were shaped in a form of arrows. Unfortunately, assuming that it worked as a hint for the users it can’t be properly justified as being the most efficient one. 
This way of non-traditional interaction was quickly familiarized by the users when encountered. This shows that in fact the knowledge gap was greatly minimized. To minimize the knowledge gap even further more iterations would need to be done.