\chapter{Discussion}
The testing of this implementation, have a few points that should change, specifically four points have been identified that has had a negative impact on the test. these are:
\begin{itemize}
\item Test environment
\item Tested users can be classified as gamers
\item Competition
\end{itemize}
these first section of the discussion will focus on explaining and solving these problems. Next the discussion will look into the main points that can be used to make the redesign better. the discussion will then look at some of the choice made along the way for which control schemes that were implemented and look at what could be alternatives. Finally the discussion chapter will look into some of the technical difficulties that the Joystick control scheme experienced(more on that later) as well as look into what could have made the test level better suited for testing navigation.


\section{Test environment}  
During the testing of this prototype the environment was not taken into account as it was done in the team's group room, this introduced some problems that should not have been present, for instance: two test participants were brought to test for the team, these were not separated during their individual tests and the testers were not told not to talk during the test. this might have affected their responses as well as their completion time. This could have been fixed by having the participants test the app in one room and do the card sorting in another. If the tests had been carried out in two rooms, one room could have held a one-way mirror where the testers could have observed the test participants from, and also filmed. this could have allowed for an in-depth analysis of the emotional responses from the participants. 
Tested users are gamers
An issue that was not taken into account before the test was carried out was the fact that many of our test participants were not only digital natives but gamers. This meant that our target group was not well represented. They may have had too many skills than we anticipated which may have affected some outcomes in the card sorting test e.g. that the buttons were boring, that they are used to using a joystick and have a smaller knowledge gap than the digital natives.

\section{Anomalies in the test data}
When looking at the data gathered from the tests, 2 test subjects stand out as anomalies. While the joystick is generally slower than the other 2 control schemes, there was one test participant who spent nearly twice as long as the second slowest test participant.
The deviation from the norm in this test was so large that it single handedly increased the average time by close to 25\%.
Another anomaly we noticed was on the gyroscope. One of our test participants ran into some technical issues during the test, to which we could attribute his overwhelmingly negative comments. This test participant accounted for 40\% of the gyroscopes negative comments.
One must note that the first of these anomalies CAN be eliminated since it's part of our quantitative data, while the other can not since it's qualitative. Even though it accounts for 40\% of the negative comments relating to this control scheme
\section{Competition}
The test was designed as a competition in hopes of motivating the user to do their best i.e. as fast as they could. This worked well during the test, but it later showed that the test participants might have been too focused on the competition rather than the navigation. This could also be because they were not instructed properly before the card sorting. However, it did affect the card sorting results as we can see many chose cards like "fun" or "engaging" and reasoned it with the fact that it was fun because it was a competition. 
We might still have been able to pull this off if we had given more clear instruction to what they should base the cards on. 
\section{Test data to redesign}
In order to test whether the button controls IS actually the most efficient, another test should be done where the test participant get a chance to play around with the controls before being timed. This should eliminate or at least reduce the knowledge gap, allowing the test to focus only on efficiency instead.
While the joystick controls received mostly negative feedback there were some positive comments though, particularly on the fact that it allowed the user to move sideways instead of just back and forth. This feature is something that shall be brought into the next iteration.


\section{Control schemes choices}
\section{Technical Difficulties - choices and preparation} 
When looking at the results for the joysticks it's obvious that these were the least popular
However, there were a lot of complaints about technical issues with the joystick controls, which means it might be necessary to calibrate it and test it in the second iteration as well. 
\section{Level design - obstacles are lower than the camera height}
This resulted in collision issues which added to the times. However this was an issue for every control scheme so it's not something that should be eliminated from the test results, it's still an unnecessary annoyance which should be removed from the next iteration.