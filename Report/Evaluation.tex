\chapter{Evaluation}
The evaluation is conducted to conclude whether the problem stated in this report is being solved. The answer will be acquired when the data gained from evaluation is analysed and concluded upon.
This section will attempt to evaluate the control schemes and will cover methods and different approaches planned to be used when conducting the evaluation. 
It will be discussed why usability testing and performance testing methods were chosen for evaluation of the prototype and why it seemed the most efficient and suitable choice. 

\section{Usability testing}

This testing method can reveal various errors that could occur when a user interacts with a product.
Usability testing is a method that seeks to test five areas. \cite{usability}

\begin{itemize}
\item Efficiency
How quickly can the user complete certain tasks when they know the design. \cite{usability}
\item Satisfaction
How satisfied are the users with the design. \cite{usability}
\item Errors
How many errors do the users make. Is it severe errors and how quickly will they recover. \cite{usability}
\item Learnability
How easy is it for the user to complete certain tasks when they first encounter the design.\cite{usability}
\item Memorability
How well do the user remember the design after some time away from the design. \cite{usability}
\end{itemize}

The benefits from usability testing is that it identifies major usability issues from a few number of participants. \cite{usability}
According to Jakob Nielsen it is enough to have only five participants as the problems will show clearly based on this alone. \cite{usability}

There are several ways to conduct a usability test e.g. focus groups, user testing etc.
The following will describe the methods we chose to carry out.

\subsection{Performance testing}

Performance testing is a method that focusses on, as the name reveals, performance. This can be measured in different ways. For instance, time and numbers of errors made.
It allows the facilitators to obtain measures of effectiveness and efficiency. \cite{performance}

The benefits of performance testing is that it reveals major problems including problems related to the users skills and expectations. \cite{performance}
Tell the user how to achieve the goal but not how to do it \cite{} 
Observe and measure without commenting\cite{}

\subsection{Card Sorting}
Card sorting is a method for discovering latent structured. \cite{cards}
It allows the test participant to give critical feedback without having to do it directly to the testers. 
You will need fifteen participants for a card sorting test \cite{cardSorting}. Normally a usability test only need five users but this testing method needs more participants to get a full view of the users preferred structure. This can not be accomplished by five participants. \cite{cardSorting}
As Jakob Nielsen says \cite{cardSorting}, this test differs from other usability tests by being a generative method. This means that we do not yet have a design and need to establish user needs first as we tested for the navigation before we designed the actual app. 

According to Jakob Nielsen, the classic way to ruin a card sorting test is to give the user familiar command names. This will make the user look for that specific command name instead of acting as they normally would.\cite{cardSorting}
E.g. Do not say "now you will use a joystick" and thus giving them the information that this should be controlled as a joystick and the usability problems there might be will not be revealed. It will not be certain if the test participant actually came to the conclusion that it should work as a joystick
themselves. 

Card sorting can be used in various ways. To group words, to name groups of words, to describe a product and more. 
We used it to make the test participant feel more comfortable choosing a critical word rather than interviewing them. 
Here is how we used it together with the performance test.

\section{First test - Navigation}

The first thing that is tested is the navigation. This needed to be established before making the actual design and to make sure that the app is usable.
It is a usability test that will consist of two parts; performance and sort-carding. 
The test is set in a controlled environment where the participants need to go from A to B, testing the control schemes, rotating the order of what control scheme they start out with.

Before the test starts, the test participants will watch us go through the level so they know where to go. This should eliminate some bias when it comes to getting to know where to go and should help keep the focus on how to get there. Otherwise the first control scheme would be slower every time as they would have to find their way through the level.

We will time them to see which control scheme is the easiest to control. We will also observe how much they struggle and how fast they get from A to B. Also we will estimate how long time it takes them to have a grip on the controls.

After each control scheme we will ask them to do a card sorting. They pick 5 words and afterwards they would be asked why they chose those particular words. This will help us getting them to say something critical about the controls to us. 

We will make a scoreboard so the test participants can see how fast they completed the course compared to the other people who tested it.
This should add a competitive element and some fun to the test.

\subsection{Analysis of data}

The first thing that was done after testing, was color coding the cards. This way we could sort them out into categories. 
Then we made bar charts to get a visual and clear image of the results from the card sorting.
The cards were sort after:

\begin{itemize}
\item Intuitive/familiar UX
\item Positive feelings
\item Negative feelings
\item Complaints
\item Unusable. These cards were not related to the control schemes.
\end{itemize}

\begin{figure}[H]
\centering
\includegraphics[scale=0.5]{Gyroscope.png}
\caption{Categorisation of the cards for the gyroscope control scheme.}
\end{figure}

The gyroscope turned out to be very intuitive and quite fun for users. It has a small amount of negative feelings and a few complaints. These were mostly based on the fact that you have to turn all the way around and not being able to sit down while using this and that it moved to slow. 
The intuitive cards that were selected showed that the users thought it was nice that their movements was met in a virtual world. Also they thought that it was easy to understand once you have practised a bit. 
The positive feelings towards the gyroscope was quite clear; it was fun, exciting and friendly. Almost all positive cards had something to with the fact that this was new and that the users had not yet experienced this which made it more attractive.

\begin{figure}[H]
\centering
\includegraphics[scale=0.5]{Buttons.png}
\caption{Categorisation of the cards for the button control scheme.}
\end{figure}

The button control scheme has the absolute highest number of intuitive/familiar feelings. This makes sense since it is common for users to use buttons. 
It has about the same amount of negative feelings and complaints as the gyroscope. These feelings were however quite interesting. 5 out of 9 negative cards said dull or boring. The users found that this was to common and had been seen before. It was clearly the most efficient as the intuitive and positive feelings were all directed towards the speed, how easy it was to use and that it was trustworthy. 
There is no way we can actually improve on the negative remarks because if the buttons are too last year our hands are tied.

\begin{figure}[H]
\centering
\includegraphics[scale=0.5]{Joystick.png}
\caption{Categorisation of the cards for the joystick control scheme.}
\end{figure}

Lastly, the joystick has a lot of negative feelings and complaints. Even more than it has intuitive/familiar feelings. This clearly shows that the users did not enjoy using this controlling scheme as much as the others and was quite frustrated with it. This could also be provoked by the way we chose to implement this and the fact that it did not resemble a joystick well enough. The main issues that was pointed out to us in the negative feelings was that it was frustrating, hard to control. The users commented on the camera the most. Also 8/18 complaints was directed towards the cameras movement and that the joystick lacked sensitivity. These negative feelings could maybe have been avoided if the implementation had been better. %reformulate this.
The positive feelings however was flexible and creative. The users thought it was fun, once you get to know it. 
The few that did give this control scheme a intuitive/familiar card thought that it was very clear that this was a joystick but still struggled to control it due to the cameras movements.

The joystick might not be as bad as it appears in the bar chart. The main reason for the high number of complaints and negative feelings was that the cameras movement was out of control. This could have been prevented by better implementation. So as far as the joystick goes, we will not know if this is the reason for the feedback until we can fix it and test it out once more.


The time from the performance tests was calculated to find the mean and made graphs to demonstrate this as well. 

\begin{figure}[H]
\centering
\includegraphics[scale=0.5]{Averagetime.png}
\caption{The average time it took for the participants to complete the level within the different control schemes.}
\end{figure}

It is fairly easy to conclude that the buttons were the most efficient and the joystick the most problematic for our users based on the time alone. But is the buttons the most successful after all?

It shows very clearly that the users had most intuitive/familiar comments about the buttons scheme. And the one they had the hardest time coping with was the joystick. 
Both the performance test and the card sorting test showed that the fastest, most intuitive and most efficient was the buttons scheme. However, their emotional response to this was not as we hoped. 
The one controlling scheme the users found the most exciting was the gyroscope. They did have some complaints, but these were mostly based on the fact that there was a learning curve and some technical issues that can be fixed. 

If we were to compare the gyroscope and the button scheme for final conclusion the users had 58 positive feelings/intuitive about the button scheme and 17 negative feelings/complaints. Where as the gyroscope had 54 positive feelings/intuitive and 15 negative feelings/complaints. 

\section{Conclusion of results}

In conclusion, we now know that we will not be using the joystick. Whether we choose the gyroscope or the buttons ultimately depends on our focus. Do we want a control scheme that is exiting, new and fun or do we want one that we know for sure will cause the users no usability problems and be the most efficient although they think it is dull? 
The final problem statement is interested in making an application that minimizes the knowledge gap using familiarity. %make labelref here.
Also, the user experience section suggest that when people think something is fast, it is good UX. \pageref{EvalConFast} And furthermore, it is concluded that we want an app that is easy to remember and learn how to use. \pageref{EvalConUsability}

Based on this, we can conclude that the buttons is the optimal choice for our navigation app. Even though the users were more emotionally engaged in the gyroscope, the focus for this app lies within familiarity and not excitement. And even though the negative cards that were chosen in the card sorting suggested that the users found this boring, it still sets the bar high with 43 cards chosen on familiarity and intuition.
 

