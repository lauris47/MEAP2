\section{Evaluation}
This section will aim to evaluate if we managed to answer our fps. the control schemes somethingsomething

\subsection{Usability testing}

This testing method can reveal various errors that could occur when a user interacts with a product.
Usability testing is a method that seeks to test five areas. \cite{usability}

\begin{itemize}
\item Efficiency
How quick can the users perform tasks when they have encountered the design. \cite{usability}
\item Satisfaction
How satisfied are the users with the design. \cite{usability}
\item Errors
How many errors do the users make, how severe are the errors and how much time will it take for them to recover. \cite{usability}
\item Learnability
How easy is it for the test participants to complete the task their given at the first encounter. \cite{usability}
\item Memorability
How easy will users re-establish the knowledge they have of a design after coming back to it after some time. \cite{usability}
\end{itemize}

According to Jakob Nielsen it is enough to have only five participants as the 75-80\% problems will show based on this alone. \cite{usability}
An important thing to remember when you are designing a usability test is to tell the user how to achieve the goal but not how to do it \cite{usability} If you do not let the user do it own their own you are comtaminating the results. \cite{usability}
There are several ways to conduct a usability test e.g. focus groups, user testing etc.
The following will describe the methods we chose to carry out.

\subsubsection{Performance testing}

Performance testing is a pretty straightforward method that focusses on, as the name reveals, performance. This can be measured in different ways. For instance, time and numbers of errors made.
It allows the facilitators to obtain measures of effectiveness and efficiency. \cite{performance}

The benefits of perfomance testing is that it reveals major problems including problems related to the users skills and expectations. \cite{performance}
In a performance test it is important to observe and measure without commenting\cite{performance}


\subsubsection{Card Sorting}
Method for discovering latent structured. \cite{cards}

Allows the test participant to give critical feedback without having to do it directly to the testers. 
You will need fifteen participants for a card sorting test \cite{cardSorting}. Normally a usability test only need five users but this testing method needs more participants to get a full view of the users preferred structure. This can not be accomplished by five participants. \cite{cardSorting}

As Jakob Nielsen says \cite{cardSorting}, this test differs from other usability tests by being a generative method. 
We do not yet have a design and need to establish user needs first. Test for the navigation before we design the actual app. 

note: Do not ruin the test by given familiar command names. They will look for that specific keyword instead of doing the task 
as the normally would. \cite{}
E.g. Do not say "now you will use a joystick" thus giving them the information and the usability problems there might be
will not be revealed. It will not be certain if the test participant actually came to the conclusion that it should work as a joystick
themselves. 

\subsection{First test - Navigation}

The first thing that is tested is the navigation. This needed to be established before making the actual design and to make sure that the app is usable.
It is a usability test that will consist of two parts; performance and sort-carding. 
The test is set in a controlled environment where the participants need to go from A to B, testing the control schemes, rotating the order of what control scheme they start out with.

Before the test starts, the test participants will watch us go through the level so they know where to go. This should eliminate some bias when it comes to getting to know where to go and should help keep the focus on how to get there. Otherwise the first control scheme would be slower every time as they would have to find their way through the level.

We will time them to see which control scheme is the easiest to control. We will also observe how much they struggle and how fast they get from A to B. Also we will estimate how long time it takes them to have a grip on the controls.

After each control scheme we will ask them to do a card sorting. They pick 5 words and afterwards they would be asked why they chose those particular words. This will help us getting them to say something critical about the controls to us. 

We will make a scoreboard so the test participants can see how fast they completed the course compared to the other people who tested it.
This should add a competitive element and some fun to the test.

\subsubsection{Analysis of data}

The first thing that was done after testing, was color coding the cards. This way we could sort them out into categories. 
Then we made bar charts to get a visual and clear image of the results from the card sorting.

\begin{figure}[H]
\centering
\includegraphics[scale=0.5]{Gyroscope.png}
\caption{Categorisation of cards chosen for the gyroscope scheme.}
\end{figure}

\begin{figure}[H]
\centering
\includegraphics[scale=0.5]{Buttons.png}
\caption{Categorisation of cards chosen for the button scheme.}
\end{figure}

\begin{figure}[H]
\centering
\includegraphics[scale=0.5]{Joystick.png}
\caption{Categorisation of cards chosen for the joystick scheme.}
\end{figure}

write something about this ...

The time from the performance tests was calculated to find the average time and to visualize this graph was made as well. 

\begin{figure}[H]
\centering
\includegraphics[scale=0.5]{Averagetime.png}
\caption{The average time it took for the participants to complete the level within the different control schemes. }
\end{figure}

It is fairly easy to conclude that the buttons were the most efficient and the joystick the most problematic for our users. 


