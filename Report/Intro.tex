\chapter{Introduction}
%The aim of our project is to give users a unique virtual experience of walking around their own designed homes in 3d. this will be  accomplished by using the existing in-built smart device sensors. A feature that is accessible in almost every modern smart device will be used - the gyroscope (along with other sensors that might compliment the product). Users will be able to experience their preferred designs in a 3d environment, being able to explore it with an additional feeling of Immersion.
%With a fast and busy lifestyle, it is hard not to think about time efficiency, especially with the tasks that people do not want to spend too much of their resources on. This is why it is important to establish pleasant experiences. The application that will be developed will help people to save not only time but expenses too. 
The goal of this project is to create an application which allows the user to navigate in a 3d environment. The intention is that the application will be used as part of an interior design process, where shoppers can set up the environment, add the décor and experience the room fully decorated, before actually paying for the furniture and setting it up.
The navigation-application will be developed with a focus on a pleasant user experience and intuitive navigation. As such we will be attempting several non-traditional interaction methods and through focus groups evaluate which methods provide the user with the most intuitive experience. 
During the early brainstorming sessions the idea was to make an application that would help real-estate agents showcase the houses that they were selling. Looking into this however showed that there was no need for this since what this application does would mostly take away from the need for a real estate agent and would likely result in a target group unwilling to cooperate. Instead we chose to focus on shoppers at IKEA. The reason for this being that it's a discount furniture store where people often go to furnish a whole room or apartment/house.
In addition, IKEA has already made a similar application for interior design which shows that there is a market for this.
 
\section{Initial Problem Statement}
How can we improve user experience in an interior design app using non-traditional mobile sensors?
