\chapter{Introduction}
%The aim of our project is to give users a unique virtual experience of walking around their own designed homes in 3d. this will be  accomplished by using the existing in-built smart device sensors. A feature that is accessible in almost every modern smart device will be used - the gyroscope (along with other sensors that might compliment the product). Users will be able to experience their preferred designs in a 3d environment, being able to explore it with an additional feeling of Immersion.
%With a fast and busy lifestyle, it is hard not to think about time efficiency, especially with the tasks that people do not want to spend too much of their resources on. This is why it is important to establish pleasant experiences. The application that will be developed will help people to save not only time but expenses too. 
%The goal of this project is to create an application which allows the user to navigate in a 3d environment. The intention is that the application will be used as part of an interior design process, where shoppers can set up the environment, add the décor and experience the room fully decorated, before actually paying for the furniture and setting it up.
%The navigation-application will be developed with a focus on a pleasant user experience and intuitive navigation. As such we will be attempting several non-traditional interaction methods and through focus groups evaluate which methods provide the user with the most intuitive experience. 
%During the early brainstorming sessions the idea was to make an application that would help real-estate agents showcase the houses that they were selling. Looking into this however showed that there was no need for this since what this application does would mostly take away from the need for a real estate agent and would likely result in a target group unwilling to cooperate. Instead we chose to focus on shoppers at IKEA. The reason for this being that it's a discount furniture store where people often go to furnish a whole room or apartment/house.
%In addition, IKEA has already made a similar application for interior design which shows that there is a market for this.
As the technology becomes more advanced, our expectations of the technology increases as well. Microprocessors have become increasingly faster and smaller and according to Moore's law\footnote{Moore's law predicts that the number of transistors will double roughly every two years.} the improvements will continue. We are now at the point where you can actually fit a computer, which 10 years ago would be considered state of the art, into your pocket and make it affordable for the general population.
As the size of the hardware decreases, so does the physical limitation of the devices. There is now room for specialized sensors in the average smartphone, things like accelerometers, magnetometers and gyroscopes along with support for multi touch, all of which allow the user to interact with their device in new and non-traditional ways.
Most applications however do not employ these sensors and as such, a lot of users are not even aware of the possibilities they provide.
By using these non-traditional interaction methods one can strive to improve the user's experience by challenging them to use their devices in new and exciting ways. When the users have their preconceptions challenged, a state of flow is established and even tedious tasks may seem fun.
This report will explore some of these non-traditional interaction methods, specifically when it comes to navigation in a 3D environment.
These methods will be explored by creating an environment using Autodesk Maya, Adobe Photoshop and Unity3d where the users will have a chance to try out various means of navigating through the three dimensional world, in an attempt to determine the most efficient and most intuitive controls.

\section{Initial Problem Statement}
How can we improve user experience when navigating a 3D space using non-traditional mobile sensors?