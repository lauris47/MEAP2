\section{Target Group}
\label{TargetGroup}
Profound understanding of the target group helps in the process of creating a useful concept. It will be an answer to the user's specific needs and wishes. Dividing the target group into different group categorizations, or segments, gives a possibility to give a deeper understanding of the chosen subjects and leads to a better concept design. Created prototypes will be specifically designed for a segmented and generalized group of people. Connecting with the target group is essential since the product will be used by actual users. In most cases it is not enough to generalize a group of people from surface observations or presumed stereotypes, as people sometimes say one, but do another, resulting in different understanding of their needs.
\\
There are many ways to segment a target audience. Probably the most popular ones are geographic or demographic segmentations. However geographic and demographic segments would give too big of a range in age, geographical position etc. so it would not give conclusive results.
\\
This project targets people who might need to use non-traditional interaction in 3D environments on their mobile devices - it can be used in many fields such as video games, simulations, controlling something that requires remote usage such as robots and planes, localization, e.g. a simulation of viewing architectural objects, landscape areas as in "Google maps" etc. This can help to analyse the target group through psychographical segment. 
\\
\subsection{Psychographics}
Psychographic generalization segments target groups according to social class, lifestyle and personality. \cite{Psychographic} It is important and relevant to understand the user's needs, their habits and personality since it can partially answer how the app's concept can be developed. It is known that users will need to use non-traditional interaction to navigate in 3D environment, so people will have to have certain digital skills. 
\\
\subsubsection{Digital knowledge}
In Marc Prensky's "Digital Natives, Digital Immigrants" \cite{DigitalImmigrants} article he categorizes his understanding of target groups into two segments, when it comes to understanding or learning digital technology. He categorizes them as "digital natives" or "digital immigrants". There are a few more categorizations that people use, such as "Born digital" or "Digital Settlers", so it is common to separate people into "digital knowledge" groups.
Digital immigrants are the ones who were born and grew up before the technological revolution. These people adopted the technology at a certain age or point in their lives when it was needed. 
Digital natives are the ones who grew up in the technological era, where they had access to the Internet, computers and probably experienced one or more ways of learning in a digital environment \cite{DigitalImmigrants}. Prensky notes that time will make everyone a "digital native", as everyone will be born in a world full of advanced technology. Old generalization terminology will not be suited in the future. He quotes Albert Einstein - "\textit{The problems that exist in the world today cannot be solved by the level of thinking that created them.}" Prensky later introduces "Digital wisdom" which is a more general term but more fitting in this era\cite{DigitalWisdom}. 
\\
Since the target group is so wide in demographical and geographical aspects, the only way to comprehend would be through psychographical analysis, specifically their profession and/or lifestyle. A person's profession and lifestyle can show if a specific person might use non-traditional interaction when it comes to navigating in a 3D environment. E.g. even a simple application will most likely use non-traditional interaction as the applications that were analysed in State of the Art section (\ref{SOTA}).
\\
Therefore, the people who have a need to use such an application, need to have digital wisdom already. The aim of this project is to help these people improve non-traditional way of interacting in 3D environment.
\\
\subsection{Target group conclusion}
The target group will be people with digital wisdom.