\section{Target Group}
\label{TargetGroup}
Profound understanding of the target group helps in the process of creating a useful concept. It will be an answer to the user's specific needs and wishes. Dividing the target group into different group categorizations, or segments, gives a possibility to give a deeper understanding of the chosen subjects and leads to a better concept design. Created prototypes will be specifically designed for a segmented and generalized group of people of this project. Connecting with the target group is essential since the product will be used by actual users. In most cases it is not enough to generalize a group of people from surface observations or presumed stereotypes, as people sometimes say one thing but do another. %Reformulate this?
Their actual needs can vary a lot.

However since this project is targeting people who might need to use non-traditional interaction in 3D environment on their mobile devices, it can be used in many fields such as video games, simulations and localization, e.g. a simulative approach of viewing architectural objects, a city and landscape areas as in "Google maps". %Controlling something that requires remote usage as robots, planes etc. 
The target group can also be very wide in all perspectives. This means that the target group is not generalizable and therefore it does not require a deeper analysis. 

There are many ways to segment a target audience. Probably the most popular is a geographic or demographic segmentation. However it is most relevant to analyse psychographics as geographics and demographics would give too big a range in age, geographical position etc. This kind of data would not be conclusive.

\subsection{Psychographics}
Psychographic generalization segments target groups according to social class, lifestyle and personality characteristics. \cite{Psychographic} It is important and relevant to understand the customer's needs, their habits and personality since it can partially answer how the app's concept can be developed.

\subsubsection{Digital knowledge}
In Marc Prensky's "Digital Natives, Digital Immigrants" \cite{DigitalImmigrants} article he categorizes his understanding of target groups into two segments, when it comes to understanding or learning digital technology. He categorizes them as "digital natives" and "digital immigrants". There are a few more categorizations that people use, such as "Born digital" or "Digital Settlers", so it is common to separate people into "digital knowledge" groups.
Digital immigrants are the ones who was born and grew up before the technological revolution, e.g. a 65 year old man who did not have the computers, digital tools or equipment that people do now. This person adopted the technology at a certain age or point in their life when it was needed. 
Digital natives are the one who grew up in the technological era, where they had access to the internet, computers and probably experienced one or more ways of learning in a digital environment \cite{DigitalImmigrants}. However, Prensky notes that time will make everyone a "digital native", as everyone will be born in a world full of advanced technology. Old generalization terminology will not be suiting in the future. He quotes Albert Einstein - "\textit{The problems that exist in the world today cannot be solved by the level of thinking that created them.}" Prensky later introduces "Digital wisdom" that is a more general term but fitting in this era.\cite{DigitaIWisdom}. Since the target group is so wide in demographic and geographic aspects that the only grasp \todo{grasp?} would be the physcographic aspects but focused profession and lifestyle. A person's profession and lifestyle can show if a specific person might use non-traditional interaction when it comes to navigating in 3D environment. E.g. a video gamer would most likely use one or more non-traditional approaches while playing mobile video game. Even a simple game on a smartphone might require multitouch or other sensors to control it. \todo{How do you know this? REFERENCES! - i added a "might".} 
Therefore, a person who has a need to use this, needs to have digital wisdom already or learn  Then the focus is not how to create such interaction but rather how to make it efficient, effective, easily learnable etc. %These aspects will be covered later in User Experience section.

\subsection{Target group conclusion}
It can be concluded that the majority of the target group will have digital-wisdom.