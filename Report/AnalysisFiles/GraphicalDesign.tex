\section{Graphical Design}
In graphical design, like any other design, there are many options to be considered. Colours, fonts, balance and many more factors should be chosen carefully. Most importantly, how to mix these elements together without making a mess. 
This will reflect greatly on how a design is being perceived. \cite{ColorMeaning}

\subsubsection{Colours}

Colours are not just colours when designing a brand, an app or a website. Colours are perceived in various ways and is a big part of how the design is coming  across to the user. \cite{ColorMeaning}

It is important to remember that when choosing the colour palette for a design, that how we perceive colour is very different. Also, colours can change according to what it is next to. Yellow might look different next to grey than it will next to purple for instance. \cite{Colour}

When it comes to colour psychology the truth is, it is too dependent on personal experience. There is no one right answer to which colour that represents a certain feeling. \cite{ColorMeaning}
There is many studies conducted on this matter. 
One study shows that 90\% of people make snap judgement based on colour alone. \cite{ColorMeaning} Another study shows that an intend of purchasing is linked with how a brand is perceived i.e. what kind of "personality" does the brand have?\cite{ColorMeaning}

\begin{figure}[H]
\centering
\includegraphics[scale=0.125]{color-emotion.jpg}
\caption{Overall image of how colours are generally percieved. \cite{ColorMeaning}}
\end{figure}

But all in all, the concept of the app is key. Almost every study shows that it is greatly more important to choose a colour that shows the personality of your product than picking a stereotype colour. \cite{ColorMeaning}

So how does one find the best way to coordinate different colours? Research indicates that the isolation effect is very useful.

\begin{figure}[H]
\centering
\includegraphics[scale=0.5]{isolation_effect.png}
\caption{"The sign-up button stands out because it's like a red "island" in a sea of blue." \cite{ColorMeaning}}
\label{isolationFig}
\end{figure}

Using the isolation effect will help the user have a more efficient experience because the most important feature e.g. a "sign up" button, stands out. \cite{ColorMeaning} Figure \ref{isolationFig}
Research suggests that a colour scheme that consists of analogues colors and combine it with a accent complimentary color or a tertiary color is preferred among users. \cite{ColorMeaning} 

When designing your layout it is key to keep everything simple and streamlined. 
Follow the general rules, left-to-right and top-to-bottom. Make sure the most important feature is in the top left corner where the user will look first.\cite{Sardo}
Be careful, yet not boring, when choosing a colour scheme. In general, keep the graphics clean and simple. No muss, no fuss. 