\section{Conclusion}
The targeted users are familiar with mobile devices and have digital wisdom. They are expected to be able to pick up a smartphone or a tablet and interact with it in a traditional way if the users are familiar with the interaction. What allows them to do this is the fact they feel that the interaction is intuitive. This means that the current knowledge point and the target knowledge point are identical, meaning the knowledge gap is non-existent, or that the transition through the knowledge gap is done effortlessly. Familiarity can be a powerful tool for minimizing the knowledge gap, which is what this project will be focusing on. This chapter has  analysed the state of the art and from this it can be concluded that non-traditional interaction is not commonly used. To improve upon the traditional interaction methods for navigating in a 3D environment, sensors such as gyroscope and accelerometer can provide a number of new ways of interacting with a system. 

\subsection{Final problem statement}
\textit{How does non-traditional interaction using the concept of familiarity help minimize the knowledge gap for users with digital wisdom, for navigating in a 3D environment?}