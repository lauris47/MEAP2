\section{Design Requirements}
Knowledge gap is all the knowledge that the app/programme will have to provide so that the user is confident in how the app/programme works. This is usually provided with tutorials but it can also be done using familiarity or intuition. When it comes to controlling devices, whether it is an interface or a game, the user is usually familiar with a way to control this specific environment. In a game the user usually controls a character with a keyboard and/or a mouse, depending on the games mechanics.  Respectively if the game is in a first person perspective, it would be common to control it with a keyboard for walking around and mouse to move a camera.  When using an internet browser it would be common that the tabs and any other navigational tool would be placed on the top of the browser’s window. One of the reasons these things are very similar in the way of controlling them is because it is easier for a user to switch between devices or programmes to use. Making the products similar makes the user’s experience to transfer between them way faster and more intuitive. For example Android smartphones and iPhones. The all have more or less the same graphical layout in terms of icon placement or function placement. The same goes for Windows and Macbook laptops. 
Making products of the same function it is common that the interaction and graphical layout, if such needed, is very similar because the users are already familiar with the way of interaction of the specific product. 
Therefore when creating a product when having user experience in mind, it is necessary to make the interaction familiar or intuitive for the user. 


As mentioned in the analysis part, familiarity is very related to intuitiveness. The goal for the prototype is to define which way to control the 3D virtual environment is the most familiar one for the user and also the most intuitive one. The navigation design choices were made. The familiarity concept was put into effect as the control schemes for controlling the environment had to be linked with something that the user might be familiar with already. The prototype needed to have two features to navigate in 3D virtual environment - the moving of the character and the camera movement/rotation. It came down to three options that seemed the most representative of the familiarity and intuitivity concepts. 
The first one was to control the camera with an in-built gyroscope in the tablet. This is familiar with natural way of a person looking around, by turning the direction user wishes to move. The movement forwards and backwards was implemented with on-screen buttons as these were familiar to the target group through the usual daily tasks - TV channel incrementing with forward/back buttons, more so - most computer games and tasks that require moving around with buttons. 
In the second prototype the user has to navigate using joysticks as this should be easy to learn for the users that have played console games that require navigating a character with joysticks placed on game console’s remote like Sony Playstation 4 or Microsoft Xbox One. Additionally, even for users with no previous joystick controller experience, that should not be a problem, as the control scheme is supposed to borrow the same concepts as scrolling an internet browser/editing tool window to the sides, and moving a computer mouse on the screen in the direction that is same as the device movement itself.
The third and the last one was made implementing only buttons as the way to move both camera and the character. In this case camera was moved only with arrow keys located on the screen and same for moving around - arrows indicating movement back and forth. This should be familiar with anyone that has buttons for navigation of any sorts in virtual environment. 


By implementing familiarity concept to the ways of navigation we have tangled the problem of minimizing the knowledge gap in a way that the user is being trained in a way that seems natural.


\subsection{Immediacy and Simplicity}
To communicate information in a simpler, and faster to perceive manner, the designs will be represented by concepts, that are already familiar to the user. This means, that to communicate information to the user, graphical elements will be represented as symbols, rather than text. As discussed in sections (?, ?, ?), this will further shape the familiarity and intuitivity for the application.


\subsection{Graphical element sizes and placement}
To further emphasize on our initial design requirements (ref to graphical design), the button size should be set accordingly - to ensure that the users would not have difficulties by unintentionally tapping the wrong section of the controls, individual buttons will be separated from each other and given the sizes for easy accessibility.

To enable a bigger view of the environment horizontally, the application will be built to primarily be viewed when holding the device in a landscape mode. Since the device is supposed to be held sideways and by both hands, all of the interaction should be done on the sides of the screen.

Initially to show the difference between movement and rotation buttons, they should be given different looks - shape of directional arrows as well as color indication.
