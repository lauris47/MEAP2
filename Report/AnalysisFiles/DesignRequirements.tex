\section{Design Requirements}
There are two main focus areas: familiarity and knowledge-gap. This chapter will discuss these two aspects and establish specific design requirements. 
The goal for the prototype is to define which way to control the 3D virtual environment is the most familiar one for the user. As mentioned in User Experience chapter (\ref{FamiliaritySection}), familiarity is very related to intuitiveness.
To minimize the knowledge gap for the users with digital knowledge, two conditions from the knowledge gap (section \ref{intuitiveConditions}) will be upheld through the concept of familiarity. The users are expected to understand the controls and use them with little amount of practice. Based on that, several design requirements need to be established. In general navigation has to give  positive user experience, user has to be in the “flow” state. 

\subsubsection{Technical requirements}
\begin{enumerate}
	\item Graphical User Interface
		\begin{enumerate}
		\item Controllers
			\begin{enumerate}
				\item The controllers have to be familiar to the user, 						the buttons need to stand out by using perceived 							affordance, isolation, visibility, mapping and 								consistency principles [ref 1 -Usability][ref 2 - 							Graphical design]
			\end{enumerate}
		\item Software engineering
			\begin{enumerate}
				\item Controls has to be responsive - effective [ref 3 - 					UX Conclusion]
				\item Controls has to give effective feedback [ref 4 - UX 					Conclusion]
			\end{enumerate}
		\end{enumerate}
	\item Software engineering
		\begin{enumerate}
			\item Controls has to be responsive - effective [ref 3 - UX 				Conclusion]
			\item Controls has to give effective feedback [ref 4 - UX 					Conclusion]
		\end{enumerate}	
\end{enumerate}
\subsubsection{Concept requirements}
\begin{enumerate}
	\item Navigation
		\begin{enumerate}
		 	\item Navigation has to be familiar for the user
		 		\begin{enumerate}
		 			\item The navigation should either relate to controls 					the users have used before, or relate directly to the 					real world.
		 		\end{enumerate}
		 	\item Make use of the gyroscope in context of familiarity;
		 	\item Consider graphical element placement and sizes as the 				navigation is for mobile platforms;
		 	\item 3D navigation test level;
		 		\begin{enumerate}
		 		\item Navigational level has to help to reveal each 						control schemes efficiency and effectiveness [Ref - 						Usability]
				The test level has to challenge the user to a certain 						degree in order to be able to compare the different 						designs. The level should must also make it clear to the 					user what the objective is and how they should achieve 						the goal in order to measure the effectiveness of the 						controls. The colour scheme of the level should help put 					the focus on the objective \ref {somethings} rather than 					the surroundings.
				\end{enumerate}
		\end{enumerate}
\end{enumerate}
		
				





\subsubsection{Conclusion}
These design requirements will help build the necessary prototypes that aim to minimize the knowledge gap using an interface to move in 3d virtual environment on a mobile platform.

