\section{Design Requirements}\label{DesignRequirements}
There are two main focus areas: familiarity and knowledge-gap. This chapter will discuss these two aspects and establish specific design requirements. 
The goal for the prototype is to define which way to control the 3D virtual environment is the most familiar one for the user. As mentioned in the User Experience chapter (section \ref{IntuIsFam}).
To minimize the knowledge gap for the users with digital knowledge, two conditions from the knowledge gap (section \ref{intuitiveConditions}) will be upheld through the concept of familiarity. The users are expected to understand the controls and use them with a small amount of practice. Based on that, several design requirements need to be established. In general, navigation has to give a positive user experience and the user has to be in a state of "flow" (section \ref{FlowTheory}). 

\subsubsection{Technical requirements}
\begin{itemize}
	\item Graphical User Interface
		\begin{itemize}
		\item Controllers
			\begin{itemize}
				\item The controllers have to be familiar to the user and the buttons needs to stand out by using perceived affordance, isolation, visibility, mapping and consistency principles (sections \ref{Usability} and \ref{GraphicalDesign}).  
				\item Consider graphical element placement and sizes as the navigation is for mobile platforms
			\end{itemize}
		\end{itemize}
	\item Software engineering
		\begin{itemize}
			\item Controls have to be responsive and effective (section \ref{EvalConUsability})
			\item Controls have to give effective feedback (section \ref{EvalConUsability})
		\end{itemize}	
\subsubsection{Concept requirements}
	\item Navigation
		\begin{itemize}
		 	\item Navigation has to be familiar for the user \\
The navigation should either relate to controls the users have used before or relate directly to the real world.	
		 	\item Make use of the gyroscope in context of familiarity
		 	\item 3D navigation test level\\
The level has to help reveal each control scheme's efficiency and effectiveness (section \ref{Usability})
The level also has to challenge the user to a certain 	degree in order to be able to compare the different designs. The level must also make it clear to the user what the objective is and how they should achieve the goal, in order to measure the effectiveness of the controls. The colour scheme of the level should help put the focus on the objective (see section \ref{Colours} for colour schemes) rather than the surroundings.
				\end{itemize}
\end{itemize}