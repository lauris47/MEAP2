\chapter{Conclusion}
There is so much new technology to keep up with if one does not want to feel left behind. It may not always be easy to learn new things and switch habits - especially when it comes to replacing old ones. This project aimed at making the transition from one way of doing things to another as smooth as possible. The mobile devices nowadays offer much more functionalities that can be applied in "unusual" interesting ways to support the problem area which was to determine the most efficient and most intuitive way of controls, the initial problem was stated: “\textit{How can we improve user experience when navigating in a 3D space using non-traditional mobile sensors?}”
The analysis gave insight into the nature of the concepts such as digital wisdom, familiarity, usability and non-traditional interaction methods. After conducting the analysis  a  more specific problem was formulated: “\textit{How does non-traditional interaction using the concept of familiarity help minimize the knowledge gap for users with digital wisdom, for navigating in a 3D environment?}”. From this a set of design requirements was established for the prototypes. From these requirements three control schemes were designed with a focus on being familiar to the users. These three controls schemes are the joystick, buttons and gyroscopic controllers, all of them developed inside the Unity 3D engine. The joystick and gyroscopic controllers both improve upon the existing state of the art by having control schemes that are not seen in the current apps.
For these three control schemes a test was designed that was designed to show the efficiency of the schemes. The test was designed around a test level that would have the user run through an environment that would force them to make use of the controllers. This environment was designed to be useful for later tests. The evaluation of the controllers showed that all three of them had strong points but in relation to the FPS it was found that the most familiar and most efficient control scheme was the buttons scheme. The testing also showed that making the users interact with digital systems via real life movement increased the emotional responses of the users. The testing was however flawed as the joystick controller did not have the same fidelity as the other two control schemes, because of this the control scheme was the scheme that showed the worst completion time on average. With the test data evaluated, a set of improvements and additions to each of the controllers was made that should be carried over into the second iteration, where the same controllers with improvements would be tested against a set of new controllers, this would help in determining if the original three controllers were actually the most efficient ones. As mentioned in the discussion chapter the users were generally quick to understand how the control schemes worked, while this would seem to support the idea that familiarity helps in minimizing the knowledge gap, it requires further testing to see if that was actually the case.